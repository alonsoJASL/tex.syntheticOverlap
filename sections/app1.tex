\chapter{El parámetro $B\slash A$}\label{section.app.ba}
El cociente $B\slash A$ tiene su origen al efectuar la expansión de Taylor de la presión en un medio dada por la ecuación de estado termodinámico $P = P(\rho, s)$, donde $\rho$ es la densidad del medio y $s$ la entropía. La notación utilizada fue empleada por primera vez por Fox y Wallace \cite{nonlinear}, quienes puntualizaron que la expansión debe considerar condiciones adiabáticas del medio. Su efecto lo vamos a encontrar en el coeficiente de no linealidad:
\begin{equation*}
\beta = 1 + \frac{B}{2A}\text{.}
\end{equation*}
\section{Definiciones}
La derivación del parámetro se tomó en cuenta también en el capítulo \ref{chapter.modelos}, retomando la ecuación de estado $P(\rho,s)$ en el medio \textbf{isoentrópico} da:
\begin{equation}
P - P_0 = \left( \frac{\partial P}{\partial \rho}\right)(\rho - \rho_0) + \frac{1}{2!}\left( \frac{\partial^2 P}{\partial \rho^2}\right)(\rho - \rho_0)^2 \label{app1.eq1}\text{,}
\end{equation}
donde $s$ es la entropía específica, las derivadas parciales se evalúan en el estado $(s_0,\rho_0)$, los valores iniciales o de ambiente del medio. Para comprender el nombre dado al parámetro, se toma la ecuación \eqref{app1.eq1}, y se expresa de la siguiente manera: 
\begin{equation}
p = A\left(\frac{\rho'}{\rho_0}\right) + B\left(\frac{\rho'}{\rho_0}\right)^2 + C\left(\frac{\rho'}{\rho_0}\right)^3 + \cdots \label{app.eq2}
\end{equation}
donde $p=P-P_0$ es la presión sonora, $\rho' = \rho-\rho_0$ es el excedente de densidad y:
\begin{subequations}
\begin{align}
A = \rho_0\left( \frac{\partial P}{\partial \rho}\right) \equiv \rho_0c_0^2 \text{,}\label{app1.eq3-1} \\
B = \rho_0^2\left( \frac{\partial^2 P}{\partial \rho^2}\right)\text{,} \label{app1.eq3-2} \\
C = \rho_0^3\left( \frac{\partial^3 P}{\partial \rho^3}\right)\text{.} \label{app1.eq3-3}
\end{align}\label{app1.eq3}
\end{subequations}
Notar que la velocidad del sonido para señal pequeña se define en la ecuación \eqref{app1.eq3-1}. Como se mencionó anteriormente, el supuesto de que las condiciones del medio son adiabaticas, implica que no hay intercambio de calor significativos que provoquen variaciones en la entropía por conducción de calor. \medskip\\
Tomamos, entonces, el cociente $B\slash A$ definido como:
\begin{equation*}
\frac{B}{A}  = \frac{\rho_0^2\left( \frac{\partial^2 P}{\partial \rho^2}\right)}{\rho_0\left( \frac{\partial P}{\partial \rho}\right)} = \frac{\rho_0}{c_0^2}\left( \frac{\partial^2 P}{\partial \rho^2}\right)\text{.}
\end{equation*}
\section{Interpretación física de $B\slash A$}
La importancia de este parámetro radica en su efecto en la velocidad de sonido, dada la relación $c^2 = (\partial P\slash \partial \rho)$ obtenemos de la ecuación \eqref{app.eq2}:
\begin{equation}
\frac{c^2}{c_0^2} = 1 + \frac{B}{A}\left(\frac{\rho'}{\rho_0}\right) + \frac{C}{2A}\left(\frac{\rho'}{\rho_0}\right)^2+ \cdots \label{app1.eq4}\text{.}
\end{equation}
Tomando raiz cuadrada y haciendo una expansión binomial nos queda: 
\begin{equation}
\frac{c}{c_0} = 1 + \frac{B}{2A}\left(\frac{\rho'}{\rho_0}\right) + \frac{1}{4}\left[ \frac{C}{A} - \frac{1}{2}\left(\frac{B}{A} \right)^2 \right]\left(\frac{\rho'}{\rho_0}\right)^2 + \cdots \label{app1.eq5}\text{.}
\end{equation}
Cuando se hace análisis lineal de amplitud pequeña, la velocidad del sonido se considera constante. Como se vio en el análisis \eqref{app1.eq5} el parámetro $B\slash A$ determina la importancia relativa del desfase entre la velocidad del sonido en un momento dado con la velocidad de sonido referente a los análisis lineales de señal pequeña $c_0$.\medskip\\
El coeficiente de no linealidad $\beta = 1 + B\slash (2A)$, utilizado en este trabajo, es la medida significativa de no linealidad acústica en el medio. Los valores que se calculan de este coeficiente para cada uno de los medios dependen de la siguiente relación:
\begin{equation}
\displaystyle\beta = \left\lbrace
\begin{matrix}
(\gamma + 1)\slash2 & \text{gases} \\
1 + B\slash (2A) & \text{líquidos}\\
-\left( \frac{3}{2} + \frac{\mathcal{A} + 3\mathcal{B} + \mathcal{C}}{\rho_0 c_1^2}\right) & \text{sólidos}
\end{matrix}
\right.
\end{equation}
donde $c_1$ es la velocidad longitudinal de onda de señal pequeña, y $\mathcal{A},\mathcal{B}, \mathcal{C}$ son las constantes elásticas de tercer orden definidas por Landau y Lifschitz (1986)\cite{nonlinear}. Los valores para $\beta$ se consultan en tablas que han hecho las estimaciones. 